\documentclass[14pt, aspectratio=169]{beamer}
\usepackage{lmodern}
\usepackage[ngerman]{babel}
\usepackage[utf8]{inputenc}
\usepackage[T1]{fontenc}
\usepackage{tikz}
\usepackage{graphicx}
\usepackage{listings}
\usepackage{dirtree}
\usepackage{courier}

\setbeamertemplate{navigation symbols}{} % Keine Navigationsleiste
\usetheme{Antibes}

% Title page information
\title{OpenCarve --- Bild-zu-G-Code Umwandlung und 3D-Visualisierung}
\author{Martin Winter}
\institute{3D-Prototyping}
\date{\today}

\begin{document}

\frame{\titlepage}

\begin{frame}
  \frametitle{Inhalt}
  \tableofcontents
\end{frame}

\section{Einführung}
\begin{frame}
  \frametitle{Einführung}
  \begin{itemize}
    \item OpenCarve wandelt Graustufenbilder in G-Code um.
    \item Der G-Code kann direkt an CNC-Maschinen übertragen werden.
    \item Zusätzlich bietet das Tool eine 3D-Visualisierung des Toolpaths.
  \end{itemize}
\end{frame}

\section{Verwendete Technologien}
\begin{frame}
  \frametitle{Verwendete Technologien}
  \begin{itemize}
    \item \textbf{Python:} Als Hauptprogrammiersprache zur schnellen Entwicklung.
    \item \textbf{PyQt5:} Für die Erstellung der grafischen Benutzeroberfläche.
    \item \textbf{OpenGL:} Für die 3D-Visualisierung des G-Codes.
    \item \textbf{NumPy \& Pillow:} Für Bildverarbeitung und numerische Berechnungen.
  \end{itemize}
\end{frame}

\section{Bedienungsablauf}
\begin{frame}
  \frametitle{Bedienungsablauf}
  \begin{enumerate}
    \item \textbf{Bild laden:} 
      \begin{itemize}
        \item Klicken Sie auf \texttt{Load Image}, um ein Graustufenbild zu laden.
      \end{itemize}

    \item \textbf{Parameter einstellen:}
      \begin{itemize}
        \item Stellen Sie Parameter wie Pixelgröße, maximale Tiefe, Safe Z, Vorschubgeschwindigkeit, Spindeldrehzahl, Step-Down und Boundary Margin ein.
      \end{itemize}
    \item \textbf{G-Code generieren:}
      \begin{itemize}
        \item Mit \texttt{Generate G-Code} wird der Code erstellt und in einem separaten Panel angezeigt.
      \end{itemize}
    \item \textbf{G-Code kopieren und speichern:}
      \begin{itemize}
        \item Nutzen Sie die Copy- und Save-Buttons, um den Code in die Zwischenablage zu kopieren oder als Datei zu speichern.
      \end{itemize}
  \end{enumerate}
\end{frame}

\section{Parameterübersicht}
\begin{frame}
  \frametitle{Parameterübersicht}
  \begin{itemize}
    \item \textbf{Pixel Size:} Bestimmt den Maßstab des Bildes in mm.
    \item \textbf{Max Depth:} Legt die maximale Schnitttiefe fest.
    \item \textbf{Safe Z:} Definiert die Höhe, bei der schnelle (rapide) Bewegungen stattfinden.
    \item \textbf{Feed Rate XY und Z:} Bestimmt die Geschwindigkeit der Maschine (XY und Z).
    \item \textbf{Spindle Speed:} Gibt die Drehzahl der Fräse an.
    \item \textbf{Step-Down:} Legt die Schrittweite fest, in der die maximale Tiefe erreicht wird.
    \item \textbf{Boundary Margin:} Rand um den Arbeitsbereich.
  \end{itemize}
\end{frame}

\section{G-Code Postprozessor}
\begin{frame}
  \frametitle{G-Code Postprozessor}
  \begin{itemize}
    \item Optimiert den generierten G-Code.
    \item Fasst aufeinanderfolgende G1-Befehle zusammen, sofern diese dieselben Parameter besitzen.
    \item Dies führt zu einer kleineren Dateigröße und effizienterer Ausführung.
  \end{itemize}
\end{frame}

\section{Fazit und Ausblick}
\begin{frame}
  \frametitle{Fazit und Ausblick}
  \begin{itemize}
    \item OpenCarve bietet einen intuitiven Workflow: Bild laden $\rightarrow$ Parameter einstellen $\rightarrow$ G-Code generieren.
    \item Die 3D-Visualisierung ermöglicht eine Vorabkontrolle der Toolpaths.
    \item Der Postprozessor sorgt für einen optimierten und maschinenfreundlichen G-Code.
    \item Zukünftige Erweiterungen können zusätzliche Funktionen und erweiterte Simulationsmöglichkeiten umfassen. STL-Support
  \end{itemize}
\end{frame}

\end{document}
